\documentclass[12pt,twoside]{book}

\usepackage[a4paper,inner=3.5cm,outer=2.5cm,top=2.5cm,bottom=2.5cm]{geometry}
\usepackage{fontspec}
\usepackage{polski}
\usepackage[polish]{babel}
\usepackage{setspace}
\usepackage{fancyhdr}
\usepackage{titlesec}
\usepackage{array} % Rozszerzenie możliwości tabel
\usepackage{lscape} % Do ustawienia tabeli na szerokość strony (jeśli jest potrzeba)
\usepackage{tabularx} % Pakiet do automatycznego dostosowywania szerokości kolumn
\usepackage{longtable} % Pakiet do tabel rozciągających się na więcej niż jedną stronę

\setmainfont{Times New Roman}
\setstretch{1.5}
\pagestyle{fancy}
\fancyhf{}
\fancyfoot[CE,CO]{\thepage}
\renewcommand{\headrulewidth}{0pt}
\fancypagestyle{plain}{%
    \fancyhf{}%
    \renewcommand{\headrulewidth}{0pt}%
    \fancyfoot[CE,CO]{\thepage}%
}


% Konfiguracja nagłówków
% Nagłówek 1. stopnia: 12 pkt, WERSALIKI, pogrubiona
\titleformat{\chapter}[hang]
{\normalfont\bfseries\fontsize{12}{14}\selectfont\uppercase}
{\thechapter.}
{1em}
{}

% Nagłówek 2. stopnia: 10 pkt, pogrubiona i kursywa
\titleformat{\section}
{\normalfont\bfseries\itshape\fontsize{10}{12}\selectfont}
{\thesection}
{1em}
{}

% Nagłówek 3. stopnia: 10 pkt, kursywa
\titleformat{\subsection}
{\normalfont\itshape\fontsize{10}{12}\selectfont}
{\thesubsection}
{1em}
{}

% Dostosowanie odstępów dla nagłówków
\titlespacing*{\chapter}{0pt}{12pt}{6pt}
\titlespacing*{\section}{0pt}{12pt}{6pt}
\titlespacing*{\subsection}{0pt}{12pt}{6pt}

\usepackage{lipsum}
\usepackage{pdfpages}
\usepackage{tocloft}
\usepackage{cite}
\usepackage{indentfirst}
\usepackage{enumitem}
\usepackage{graphicx}
\usepackage[font=normalsize,labelfont=bf]{caption} % Konfiguracja podpisów
\usepackage[hidelinks]{hyperref}

\renewcommand{\contentsname}{Spis treści}
\renewcommand{\cftchapleader}{\cftdotfill{\cftdotsep}}
\renewcommand{\cfttoctitlefont}{\bfseries\fontsize{12pt}{14pt}\selectfont}
\renewcommand{\cftloftitlefont}{\bfseries\fontsize{12pt}{14pt}\selectfont}
\renewcommand{\listtablename}{Spis tabel}

\setlength{\parindent}{1.25cm}

\captionsetup[figure]{
    labelsep=period, % Ustawienie kropki zamiast dwukropka
    justification=centering, % Wyśrodkowanie podpisu
    font=normalsize, % Rozmiar czcionki podpisu
    textfont=normalfont, % Styl czcionki podpisu
    labelfont=normalfont, % Pogrubienie etykiety "Rys."
    name=Rys., % Zmiana nazwy "Rysunek" na "Rys."
    skip=12pt %Ustawienie dolnego odstępu podpisu
}
\captionsetup[table]{skip=6pt} % Górny odstęp dla podpisu tabeli
\captionsetup[table]{position=above} % Pozycja podpisu nad tabelą
\captionsetup[table]{justification=raggedright, singlelinecheck=false, labelsep=period}

\setlength{\cftbeforelottitleskip}{10pt} % Odstęp przed tytułem spisu tabel
\setlength{\cftafterlottitleskip}{10pt}  % Odstęp po tytule spisu tabel
\setlength{\cftbeforeloftitleskip}{10pt} % Jeśli używasz również spisu rysunków

\renewcommand{\cfttabfont}{\normalsize} % Styl numeru tabeli w spisie
\renewcommand{\cfttabpagefont}{\normalsize} % Styl numeru strony w spisie
\renewcommand{\cftlottitlefont}{\bfseries\fontsize{12pt}{14pt}\selectfont} % Styl tytułu spisu tabel
\renewcommand{\cftafterlottitle}{\bfseries\fontsize{12pt}{14pt}\selectfont} % Wyrównanie tytułu do lewej


% Dodanie odstępu przed podpisem rysunku
\newcommand{\captionvspace}{\vspace{6pt}}

% Dostosowanie odstępów dla środowiska 'itemize'
\setlist[itemize]{topsep=0pt,partopsep=0pt}

\begin{document}

    \setcounter{page}{1}
    \thispagestyle{empty}
    \includepdf{includes/pdf/strona_tytułowa.pdf}

% Tutaj umieść swoją stronę ze spisem treści
    \tableofcontents

% Tutaj zaczyna się główna treść pracy


    \chapter{Wstęp}


    \section{Cel pracy}

    Celem niniejszej pracy dyplomowej jest zbadanie procesu migracji systemu monolitycznego do architektury mikroserwisowej opartej o usługi chmury obliczeniowej dostawcy Amazon Web Services (w skrócie AWS) pod kątem ulepszenia zabezpieczeń omawianego systemu. System objęty badaniem, to: „System do internetowego wspomagania pacjenta i lekarza”, który został w całości zaprojektowany i zaimplementowany w formie monolitycznej przez autora pracy dyplomowej jako projekt inżynierski w celu ukończenia studiów inżynierskich pierwszego stopnia. Wybór systemu do przeprowadzenia analizy motywowany jest znajomością jego architektury, wpasowującą się doskonale w temat pracy oraz motywacja do jego dalszego rozwijania i ulepszania w zakresie cyberbezpieczeństwa.

    Analiza procesu migracji oraz rozwinięcia systemu zabezpieczeń systemu ma wykazać jak dobrze usługi chmurowe ochraniają swoich usługobiorców i ich oprogramowania przed popularnymi atakami hakerskimi oraz jak duży wpływ na bezpieczeństwo systemu ma jego architektura. Narzędzia używane przez autor pracy to przede wszystkim oprogramowanie wirtualizujące systemy operacyjne Docker, język programowania PHP, język programowania Java Script, zestaw narzędzi programistycznych Symfony oraz Vue.js, dodatkowo usługi AWS, takie jak: Elastic Compute Cloud (w skrócie EC2), Relational Databases (w skrócie RDS), Route 53, Code Pipeline, Code Build, Code Deploy, Elastic Cache, Simple Storage Service, Elastic Container Registry, Lambda. Również narzędzia do analizy ruchu sieciowego, takie jak: Wireshark.

    Praca ta omówia aspekty takie jak: ewaluacja zastanej architektury aplikacji, planowanie architektury mikroserwisowej w chmurze AWS, ocena zagadnień bezpieczeństwa podczas migracji, implementacja i wdrażanie mikroserwisów w chmurze AWS, ocena i analiza wyników. Każdy wymieniony etap pracy jest szczegółowo omówiony, odzwierciedlając wiedzę i doświadczenie akademickie jak i zawodowe autora.


    \section{Motywacja}

    Motywacją do podjęcia tematu pracy są doświadczenia autora na tle zawodowym, które przyspieszyły naukę w zakresie obliczeń chmurowych. Jednak największym motywatorem do wykonania badania jest kontynuacja pracy akademickiej, dążenie do ciągłego rozwoju wyprodukowanego przez siebie oprogramowania i wdrażanie wiedzy pozyskanej w czasie studiów drugiego stopnia, tak by umiejętnie zabezpieczać oprogramowanie. Dodatkowym motywatorem jest również fakt, iż technologia chmurowa każdego roku zdobywa coraz większy udział na rynku usług hostingowych, a także staje się dzięki wzrastającej konkurencyjności przystępniejsza dla mniejszych firm lub prywatnych odbiorców. Serwis Precedence Research przewiduje, iż w nadchodzących ośmiu latach udział rynkowy technologii chmurowych zwiększy się około pięciokrotnie do wartości 2297 miliardów dolarów \cite{p.research}. Patrz rysunek~\ref{fig:precedence-research}.

    \begin{figure}[ht]
        \centering
        \includegraphics[width=\textwidth]{includes/images/precedence-research.png}
        \captionvspace
        \caption{Wykres przedstawiający prognozę wzrostu udziału rynkowego technologii chmurowej}
        \label{fig:precedence-research}
    \end{figure}


    \section{Założenia i spodziewane wyniki}

    Rezultatem niniejszej pracy dyplomowej jest rozebranie monolitycznego systemu na mikroserwisy z pojedynczą odpowiedzialnością logiczną. Zasada działa aplikacji nie ulega zmianie, dla użytkownika zewnętrznego wprowadzenie zmiany powinno odbyć się bez odczuwalnej różnicy. Samo rozczłonkowanie systemu odbywa jak najbardziej atomicznie się da przy rozsądnym nakłądzie pracy pozwalającym ukończyć tę pracę w terminie jej obrony. Autor zakłada, iż poprawie ulegnie bezpieczeństwo danych przechowywanych w bazie danych poprzez ukrycie bazy danych przed dostępem z zewnątrz przy wykorzystaniu wirtualnej sieci prywatnej, także większa granularność aplikacji poprawi bezpieczeństwo wdrożenia zmian poprzez ustystematyzowanie procesu jaki za tym stoi poprzez wykorzystanie technologii „Pipeline” (ciąg zautomatyzowanych procesów dostarczania nowego oprogramowania). Również autor odświeżył wersję zależności z jakich składa się projekt, tak by rozwiązać problemy luk bezpieczeństwa wynikających z użycia przestarzałych bibliotek. Dodatkowym poziomem ulepszenia zabezpieczeń aplikacji będzie przeszukanie kodu w pod kątem znalezienia luk logicznych lub twardo zakodowanych dostępów, takich jak hasła do bazy danych lub inne, z których aplikacja korzysta.


    \chapter{Ewaluacja zastanej architektury aplikacji}

    Zgodnie z stanem aplikacji z dnia 11.06.2022r. projekt “System do internetowego wspomagania pacjenta i lekarza” oparty jest o kilka składowych jakie należy wyróżnić, a są nimi:

    \begin{itemize}
        \item baza danych oprata o silnik MySQL,
        \item trzon logiki biznesowej oparty o zestaw narzędzi Symfony w wersji 5.4,
        \item warstwa widoku aplikacji oparta o silnik Twig oraz Vue.js,
        \item silnik nginx jako narzędzie do przekazania zapytania klienckiego do logiki biznesowej.
    \end{itemize}

    Wyróżnione komponenty są ściśle ze sobą połączone. Logika biznesowa oraz baza danych są nierozerwalne, logika biznesowa posiada klasy encji, które bezpośrednio przekładają się na tabele w bazie danych. Logika biznesowa również decyduje o tym jaki widok jest obecnie wyświetlany, a warstwa widoku bazuje na tym, co dostarczył kontroler Symfony. Cały zestaw wykorzystuje narzędzie nginx do tego, by otrzymać zapytanie od klienta. W niniejszym rodziale zostaną przedstawione kolejno wyżej wymienione komponenty z szerszym opisem co do każdego z nich.


    \section{Logika biznesowa i baza danych}

    Symfony jako framework, który za zadanie ma ułatwić rozwój oprogramowania internetowego w projekcie z pracy inżynierskiej autora niniejszej pracy spełnił swoją rolę niejako przyczyniając się do skrócenia czasu potrzebnego na rzecz implementacji założeń diagramów użyć, odwzwierciedlenie domen każdego z aktorów systemu, a także diagramów przepływu danych. Narzędzia takie jak wbudowane kontrolery z dostępnym API do zapytań, czy formularze do walidacji danych wejściowych do systemu, router odpowiadający na zapytania, czy szablony wspierane przez silnik Twig to tylko niektóre z narzędzi jakie zostały wykorzystane przy okazji implementacji. Wyczerpujący schemat udogodnień oferowanych przez Symfony znajduje się na rysunku~\ref{fig:symfony-metadata-scheme}.

    \begin{figure}[ht]
        \centering
        \includegraphics[width=\textwidth]{includes/images/symfony-metadata-scheme.png}
        \captionvspace
        \caption{Schemat metamodelu Symfony \cite{mod.driv.arch.symf}}
        \label{fig:symfony-metadata-scheme}
    \end{figure}

    W celu odzwierciedlenia domen aktorów, w projekcie została wykorzystana biblioteka Doctrine, wspierana przez Symfony. Samo Doctrine to narzędzie do manipulacji bazą danych przy wykorzystaniu obiektów PHP \cite{symf.5}. By zainicjować dane w systemie bez konieczności manualnego wprowadzania ich autor wykorzystał dedykowaną ku temu bibliotekę wykorzystującą mechanizm fabryki danych testowych (ang. Fixtures Factory). Sam mechanizm fabryki danych testowych w procesie tworzenia oprogramowania umożliwia uruchomienie systemu z wypełnioną bazą danych, tak by móc przeprowadzać wymagane akcje i sprawdzić poprawność logiki biznesowej \cite{fixtures}. Fabryka kreująca dane testowe zostałą przedstawiona na rysunku~\ref{fig:example-sdiwpil-fixtures-factory}.

    \begin{figure}[ht]
        \centering
        \includegraphics[width=\textwidth]{includes/images/example-fixtures-factory-before-migration.png}
        \captionvspace
        \caption{Zrzut ekranu przedstawiający przykładową fabrykę danych testowych w Symfony}
        \label{fig:example-sdiwpil-fixtures-factory}
    \end{figure}

    Do weryfikacji poprawności danych wykorzystano wbudowany w Symfony mechanizm formularzy. Każdy formularz jest tworzony na potrzeby obsługi konkretnego obiektu encji lub specyficznego rodzaju danych, jak na przykład numer PESEL. Formularz przeznaczony do walidacji encji zawiera pola odpowiadające właściwościom tej encji oraz określa typy tych pól. Każde pole może mieć przypisany szereg ustawień, z których najważniejsze jest ustawienie przechowujące tablicę ograniczeń dotyczących wartości przekazywanych do danego pola.

    Wykorzystując ten mechanizm, można na poziomie walidacji danych przeprowadzić weryfikację, czy np. encja z podanymi danymi, które mają być unikalne, znajduje się już w bazie danych. Przykładowy formularz został przedstawiony na rysunku~\ref{fig:example-sdiwpil-form}.

    \begin{figure}[ht]
        \centering
        \includegraphics[width=\textwidth]{includes/images/example-form-before-migration.png}
        \captionvspace
        \caption{Zrzut ekranu przedstawiający przykładowy formularz w Symfony}
        \label{fig:example-sdiwpil-form}
    \end{figure}

    Do pobierania danych z bazy danych wykorzystano mechanizm repozytorium, który jest ściśle powiązany z odpowiednią encją. Przykładowo, encja \texttt{PatientData} posiada swoje repozytorium \texttt{PatientDataRepository}. Repozytorium powinno być zbudowane w sposób możliwie najbardziej generyczny, operując wyłącznie na encjach lub danych pochodzących z encji. Przykładowy kod repozytorium został przedstawiony na rysunku~\ref{fig:example-sdiwpil-repository}.

    \begin{figure}[ht]
        \centering
        \includegraphics[width=\textwidth]{includes/images/example-repository-before-migration.png}
        \captionvspace
        \caption{Zrzut ekranu przedstawiający przykładowe repozytorium w Symfony}
        \label{fig:example-sdiwpil-repository}
    \end{figure}

    Ostatnim z mechanizmów Symfony, który został wykorzystany przez autora niniejszej pracy do zbudowania pierwotnej wersji systemu do internetowego wspomagania pacjenta i lekarza, jest mechanizm walidatora. Jest to część aplikacji współpracująca z formularzami, która umożliwia niestandardową weryfikację danych. Przykładowy walidator został przedstawiony na rysunku~\ref{fig:example-sdiwpil-validator}.

    \begin{figure}[ht]
        \centering
        \includegraphics[width=0.7\textwidth]{includes/images/example-validator-before-migration.png}
        \captionvspace
        \caption{Zrzut ekranu przedstawiający przykładowe walidator w Symfony}
        \label{fig:example-sdiwpil-validator}
    \end{figure}

    Powyżej przedstawiono elementy zestawu narzędzi programistycznych Symfony, które zostały wykorzystane w budowaniu projektu. Zastosowanie tych rozwiązań pozwoliło autorowi skorzystać ze sprawdzonych mechanizmów, które stworzyły bezpieczne środowisko do implementacji założeń biznesowych. Proces implementacji z pewnością byłby trudniejszy bez ich obecności.

    \section{Warstwa widoku}

    Rozdział ten opisuje sposób implementacji warstwy widoku aplikacji. Do przeprowadzenia procesu implementajci zostały wykorzystane narzędzia takie jak:

    \begin{itemize}
        \item Twig, który jest silnikiem szablonów dla systemów opartych o język programownaia PHP,
        \item Vue.js, który dedykowany jest tworzeniu interaktywnych interfejsów użytkownika.
    \end{itemize}


    Wybór tych narzędzi został podyktowany personalnymi preferancjami autora pracy. Szablony Twig skłądają się z kilku sekcji, które zostały zdefiniowane przez autora w celu usprawnienia procesu tworzenia. Bazowy szablon przedstawiony jest na rysunku~\ref{fig:twig-base-template} i zawiera on bloki takie jak:

    \begin{figure}[ht]
        \centering
        \includegraphics[width=\textwidth]{includes/images/twig-base-template.png}
        \captionvspace
        \caption{Zrzut ekranu przedstawiający przykłady szablon Twig}
        \label{fig:twig-base-template}
    \end{figure}

    \begin{itemize}
        \item \texttt{title} - blok zawierający tytuł strony,
        \item \texttt{stylesheets} - blok zawierający odwołania do arkuszy stylów,
        \item \texttt{body} - blok zawierający treść strony,
        \item \texttt{javascripts} - blok zawierający odwołania do skryptów JavaScript.
    \end{itemize}

    Na rysunku~\ref{fig:twig-inheriting-template} przedstawiono przykładowy kod szablonu, który dziedziczy po szablonie bazowym.

    \begin{figure}[ht]
        \centering
        \includegraphics[width=\textwidth]{includes/images/twig-inheriting-template.png}
        \captionvspace
        \caption{Zrzut ekranu przedstawiający przykładowy szablon dziedziczący po szablonie bazowym}
        \label{fig:twig-inheriting-template}
    \end{figure}

    \chapter{Planowanie architektury mikroserwisowej w chmurze AWS}
    Niniejszy rozdział pokrywa zagadnienie planowania architektury mikroserwisowej oraz wykorzystanych usług potrzebnych do realizacji celu uruchomienia platformy. Nie istnieje mapa działań jakie należy wykonać by zgodnie z sztuką podzielić aplikację monolityczną na zbiór mikroserwisów. Istnieją natomiast wskazówki, którymi można się posługiwać by ułatwić doprowadzenie tego przedsięwzięcia do końca, a są nimi:

    \begin{itemize}
        \item identyfikacja mikroserwisów,
        \item dbanie o szczególną troskę w procesie ekstrakcji modułów, które są kandydatami na mikroserwisy,
        \item wprowadzenie modelu heksagonalnego aplikacji. \cite{java.ee.8.design.patterns}
    \end{itemize}


    \section{Wybór odpowiednich usług do utrzymania aplikacji}
    W wyborze usług potrzebnych do użycia dla architektury mikroserwisowej, autor korzystał z własnych doświadczeń w pracy z platformą AWS. W związku z powyższym powody wyboru konkretnej usługi zostaną przedstawione w podrozdziale poświęconym na rzecz opisania motywacji doboru.

    \subsection{Amazon VPC - Virtual Private Cloud (ang. prywatna, wirtualna chmura)}
    Powody wyboru VPC dla celu realizacji architektury sieciowej:

    \begin{itemize}
        \item możliwość tworzenia wyizolowanych sieci w chmurze, co zapewnia wysoki standard bezpieczeństwa dla systemu, pozwala kontrolować ruch sieciowy przy pomocy reguł zapory sieciowej,
        \item możliwość kontrolowania adresacji IP, podsieci, tabel routingu i bram internetowych, co pozwala na dostosowanie sieci do indywidualnych potrzeb budowanego przez autora systemu. \cite{aws.vpc}
    \end{itemize}

    \subsection{Amazon ECS - Elastic Container Service (ang. elastyczny serwis kontenerów)}
    Wybór ECS jako usługi pozwalającej na uruchomienie mikroserwisów jest uargumentowany niniejszymi powodami:

    \begin{itemize}
        \item ułatwione zarządzanie kontenerami poprzez wbudowane narzędzia do orkiestracji nimi,
        \item możliwość uruchomienia kontenerów bez konieczności zarządzania architekturą serwerową. \cite{aws.ecs}
    \end{itemize}

    \subsection{Amazon RDS - Relational Database Service (ang. zarządzany serwis relacyjnych baz danych)}
    RDS jest idealnym wyborem do zarządzania bazami danych z następujących powodów:

    \begin{itemize}
        \item automatyzuje zadania typu tworzenie kopii zapasowych, aktualizacje oprogramowania oraz monitorowanie,
        \item posiada rozwiązania gwarantujące wysoką dostępność,
        \item oferuje szyfrowanie danych w spoczynku i w trakcie przesyłania oraz przy wykorzystaniu VPC do wyizolowania usługi zapewnia wysoki poziom bezpieczeństwa. \cite{aws.rds}
    \end{itemize}

    \subsection{Amazon MQ - Amazon Managed Message Broker Service  (ang. zarządzany broker komunikatów dostarczany przez Amazon)}
    Z racji wykorzystania protokołu AMQP w projekcie migracji systemu, Amazon MQ jest odpowowiednim wyborem do zarządzania komunikacją między mikroserwisami z następujących powodów:

    \begin{itemize}
        \item umożliwia zautomatyzowanie konfiguracji, skalowania i zarządzania infrastrukturą brokerską,
        \item obsługuje popularne protokoły wiadomości, takie jak: AMQP, MQTT oraz STOMP, najbardziej kluczowym dla aplikacji systemu do internetowego wspomagania pacjenta i lekarza jest AMQP,
        \item posiada wbudowane mechanizmy redundancji i automatycznego przełączania awaryjnego,
        \item oferuje szyfrowanie danych oraz izolację środowiska w VPC. \cite{aws.mq}
    \end{itemize}

    \subsection{Amazon IAM - Identity and Access Management (ang. zarządzanie tożsamością i dostępem)}
    Najważniejszymi powodami do wyboru IAM jako usuługi zarządzającej dostępem są:

    \begin{itemize}
        \item precyzyjne zarządzanie dostępem do wszystkich zasobów AWS,
        \item umożliwienie definiowania szczegółowych polityk uprawnień dla użytkowników, grup i ról,
        \item zapewnia wysoki poziom bezpieczeństwa poprzez funkcje takie jak dwuskładnikowe uwierzytelnianie (MFA), tymczasowe poświadczenia oraz możliwość monitorowania działań użytkowników,
        \item IAM udostępnia centralne zarządzanie tożsamościami i uprawnieniami, przez co upraszcza administrację i dodaje przejrzystości zarządzania dostępem. \cite{aws.iam}
    \end{itemize}

    \subsection{Amazon S3 - Simple Storage Service (ang. prosty magazyn danych)}
    Przechowywanie danych aplikacji to kluczowe zadanie, jakie zostało postawione usłudze S3, a jej wybór został podyktowany następującymi powodami:

    \begin{itemize}
        \item usługa ta automatycznie skaluje się w górę lub dół, przez co umożliwia praktycznie nieograniczone ilościowo składowanie danych,
        \item oferuje wysoką trwałość danych na poziomie bliskim 100\%, a także wysoką dostępność,
        \item zapewnia szyfrowanie danych, zarządzanie dostępem na poziomie obiektu oraz integrację z AWS IAM,
        \item umożliwia optymalizację kosztów na podstawie częstotliwości dostępu do danych i wymagań dotyczących trwałości. \cite{aws.s3}
    \end{itemize}

    \subsection{Amazon CloudWatch}
    Usługa o nazwie CloudWatch, w szerokim spektrum usług jakie oferuje platforma AWS, dedykowana jest monitorowaniu i logowaniu akcji wykonanych w samych usługach, jak i w kodzie aplikacji. To są powody jakie zostały uwzględnione przy podjęciu decyzji o skorzystaniu z niej:

    \begin{itemize}
        \item umożliwia centralne zarządzanie i analizę danych,
        \item oferuje zaawansowane możliwości monitorowania aplikacji i infrastruktury, a także konfigurację alarmów, które powiadamiają o problemach w czasie rzeczywistym,
        \item na podstawie logów zbieranych przez CloudWatch, można zautomatyzować uruchamianie akcji skalowania zasobów, czy też funkcji Lambda. \cite{aws.cloud.watch}
    \end{itemize}

    \subsection{Amazon Route 53}
    Jako, iż system modernizowany przez autora niniejszej pracy dyplomowej jest aplikacją internetową, to należy zapewnić mu dostęp do domeny, która będzie propagowoać jego usługi na cały świat. Usługą, w portfolio AWS, która zapewnia takie rozwiązania jest Amazon Route 53. Poniżej przedstawione są najważniejsze powody, które zostały uwzględnione przy dokonaniu wyboru:

    \begin{itemize}
        \item wysoka dostępność i niezawodność gwarantowane przez rozproszenie usługi DNS,
        \item elastyczność i skalowalność,
        \item zintegrowane funkcje geolokalizacji,
        \item integracja z innymi, ważnymi z punktu widzenia systemu usługami, takimi jak: Elastic Load Balancing i CloudFront. \cite{aws.route53}
    \end{itemize}

    \subsection{Amazon Elastic Load Balancing}
    Oferowana przez firmę Amazon usługa Elastic Load Balancing (ang. elastyczne balansowanie obciążeniem) jest kluczowym elementem każdej aplikacji z uwagi na rozproszenie obciążenia systemu. Niżej przedstawiono wszystkie powody wykorzystania omawianej usługi:

    \begin{itemize}
        \item automatycznie rozkładanie ruchu sieciowego między wieloma zasobami,
        \item poprawa dostępności aplikacji poprzez monitorowanie stanu zasobów i automatyczne przekierowanie ruchu do zdrowych instancji,
        \item zarządzanie ruchem na podstawie zawartości, geolokalizacji i opóźnienia między żądanym zasobem,
        \item integracja z AWS Certificate Manager do zarządzania certyfikatami SSL/TLS, a także ochrona przez atakami DDoS i zarządzanie ruchem HTTP/2,
        \item integracja z usługą Auto Scaling (ang. automatyczne skalowanie) oraz CloudWatch. \cite{aws.elb}
    \end{itemize}

    \subsection{ACM - AWS Certificate Manager (ang. menadżer certyfikatów AWS)}
    Narzędzie ACM to natywne rozwiązanie chmury AWS oferujące szereg udogodnień istotnych z punktu widzenia cyberbezpieczeństwa, a są nimi:

    \begin{itemize}
        \item automatyzacja procesu tworzenia, wdrażania i odnawiania SSL/TLS,
        \item integracja z Elastic Load Balancing (ELB), Amazon CloudFront i API Gateway (ang. bramka API),
        \item bezpłatne utrzymanie certyfikatów SSL/TLS dla domen zarządzanych przez AWS,
        \item łatwość użycia poprzez uproszczenie procesu zarządzania certyfikatami dzięki intuicyjnemu interfejsowi. \cite{aws.acm}
    \end{itemize}

    \subsection{Amazon API Gateway (ang. Bramka API Amazon)}
    Usługa Amazon API Gateway została stworzona z myślą o maksymalnym uproszczeniu tworzenia struktury API i łączenia zasobów w jedną całość. Autor niniejszej pracy dyplomowej wykorzystał ją, po uprzednim zgłębieniu jej zalet, a są nimi:

    \begin{itemize}
        \item integracja z Amazon Elastic Load Balancing,
        \item zarządzanie ruchem poprzez mechanizm Throttiling Rules (ang. zasady tłumienia),
        \item łatwość tworzenia API i jego wdrożenia,
        \item monitoring operacji wykonywanych wewnątrz API. \cite{aws.api.gateway}
    \end{itemize}


    \section{Opracowanie planu migracji}
    Migracja działającego systemu monolitycznego na architekturę mikroserwisów jest wyzwaniem. Wiąże się tym duży koszt wstępny jaki musi ponieść przedsiębiorstwo, by uruchomić szereg koniecznych procesów pozwalających przeprowadzić odpowiednio migrację. Samą migrację należy przeprowadzić w pełnej świadomości i po przeprowadzeniu uprzednio audytu opłacalności tego zabiegu. Zabieg ten wprowadza znaczne skomplikowanie do projektu i wymaga wykwalifikowanej kadry inżynierów, którzy będą w stanie zapewnić ciągłość działania aplikacji w fazach przejściowych, a przy okazji rozwijając nową funkcjonalność.
    W niniejszej pracy dyplomowej migracja zostanie przeprowadzona na systemie hobbystycznym, a co za tym idzie jego ewentualna przerwa w działaniu nie zadziała destrukcyjnie na ewentualne przedsiębiorstwo, które mogłoby ponieść koszta niedziałającej platformy.

    \subsection{Domain-Driven Design}
    W skrócie DDD. Nazwa ta w języku polskim oznacza nacisk na prowadzenie rozwoju oprogramowania w kontekście jakiejś domeny. Jest to filozofia, która ma pomagać rozwiązywać problemy budowy oprogramowania dla skomplikowanych domen \cite{patterns.principles.and.practices.of.ddd}.
    To podejście zakłada podział systemu na komponenty oraz ich zachowania, aby te wiernie odzwierciedlały rzeczywistość. Po wstępnej fazie planowania obszarów systemu następuje przejście do fazy implementacji.

    Zadaniem autora w niniejszej pracy dyplomowej było wykorzystanie potencjału jaki został stworzony w jego pracy inżynierskiej. Mianowicie w omawianej pracy system został podzielony w naturalny sposób pośród czterech aktorów, jakimi są: pacjent, lekarz, pracownik recepcji oraz pracownik administracji. W tego podziału wyłania się domena podziału. Każdy aktor ma swój zestaw czynności jakie w swojej domenie może wykonywać. Tak na przykład każdy użytkownik ma prawa do edycji swojego konta, usunięcia go, oraz podglądu danych jakie się w nim znajdują. Przyglądając się funkcjonalności wizyt wyłania się osobna zależność, która jest możliwa do wydzielenia z całości systemu, a dostęp mają do niej aktorzy tacy jak: pacjent, lekarz i pracownik recepcji. Każdy z aktorów ma też swój indywidualny zestaw danych, który jest niezależny od innego z aktorów, co kwalifikuje ten fakt na uwzględnienie go jako podział na cztery dodatkowe domeny. Osobną zależnością jest również sama wysyłka maili, która z punktu widzenia systemu jest istotna, ale bardzo mała porównując zakres czynności jakie wykonuje.

    Dalsze działania wynikające z planowania będą bazować bezpośrednio na wytyczonym powyżej podziale. Sam podział nie jest nowy w kontekście całego systemu. Autor zgodnie z sztuką zbudował projekt, tak by w przyszłości móc go rozwinąć właśnie poprzez modyfikację w kierunku mikroserwisów.

    \subsection{Zakres migracji}
    Według autora książki “Monolith to Microservices“ należy zrozumieć w jakim zakresie szczegółowa ma być domena, którą próbuje się dzielić na mniejsze części, tak by było to rozsądne \cite{monolith.to.microservices}. Patrz rysunek~\ref{fig:ddd-sdiwpil}.

    \begin{figure}[ht]
        \centering
        \includegraphics[width=\textwidth]{includes/images/ddd-sdiwpil.png}
        \captionvspace
        \caption{Zakres domenowy systemu do internetowego wspomagania pacjenta i lekarza}
        \label{fig:ddd-sdiwpil}
    \end{figure}

    \subsection{Podejście Event Storming}
    Podejście to polega na zgrupowaniu potencjalnych domen systemu za pośrednictwem akcji jakie wykonują. Tak na przykład w systemie do internetowego wspomagania pacjenta i lekarza da się pogrupować czynności wykonywane w systemie pod kątem wysyłki wiadomości e-mail. Na (Wstaw referencję obrazka) widoczny jest schemat pokazujący, że poszczególne domeny, takie jak: użytkownicy, wizyty są zgrupowane do akcji wysyłania wiadomości drogą mailową. Całość potencjału omawianego podejścia została przedstawiona na rysunku~\ref{fig:event-storming-approach}.

    \begin{figure}[ht]
        \centering
        \includegraphics[width=\textwidth]{includes/images/event-storming-approach.png}
        \captionvspace
        \caption{Wysyłka wiadomości jest logicznie połączona dla tego modelu domeny, więc jej dalsza ekstrakcja może być trudna}
        \label{fig:event-storming-approach}
    \end{figure}


    \chapter{Przebieg migracji}
    W niniejszym rozdziale autor pracy dyplomowej przedstawił przebieg migracji, dzieląc go na osiem faz pośrednich. Przebieg ten jest odzwierciedleniem nakładu prac wykonanego przez autora na cel migracji systemu. Jest to połowa z potrzebnych działań praktycznych, ponieważ dalsza ich część to uruchomienie infrastruktury chmurowej w AWS.


    \section{Faza pierwsza}
    Po fazie planowania należy przejść do fazy, w której mając schemat podzielona zostanie całkowita migracja na fazy, tak by utrzymać ciągłość działania systemu. System uprzednio uruchomiony został w zwirtualizowanym środowisku Vagrant, gdzie miał zainstalowanie, które było proxy (ang. pośrednikiem) zapytań - nginx. W tym samym środowisku uruchomiony był również system do zarządzania relacyjną bazą danych - MySQL. Dodatkowo uruchomiona była tam również logika biznesowa przy pomocy gniazda PHP. Schematyczny obraz fazy pierwszej znajduje się na rysunku~\ref{fig:migration-phase-1}.

    \begin{figure}[ht]
        \centering
        \includegraphics[width=\textwidth]{includes/images/migration-phase-1.png}
        \captionvspace
        \caption{Schemat architektury systemu do internetowego wspomagania pacjenta i lekarza w pierwszej fazie}
        \label{fig:migration-phase-1}
    \end{figure}


    \section{Faza druga}
    Wspieranym oprogramowaniem wirtualizującym na środowisku lokalnym i w samym AWS jest Docker. W drugiej fazie migracji wykonano opisanie bazy danych, pośrednika zapytań nginx i logiki biznesowej w osobnych, niezależnych kontenerach, skomunikowanych ze sobą wirtualną siecią. Omawiana zmiana widoczna na rysunku~\ref{fig:migration-phase-2}.

    \begin{figure}[ht]
        \centering
        \includegraphics[width=\textwidth]{includes/images/migration-phase-2.png}
        \captionvspace
        \caption{Schemat architektury systemu do internetowego wspomagania pacjenta i lekarza w drugiej fazie}
        \label{fig:migration-phase-2}
    \end{figure}


    \section{Faza trzecia}
    W fazie trzeciej migracji systemu należy zacząć od domeny użytkownika i wydelegować ją do osobnego mikroserwisu od reszty monolitu, dodatkowo w samym monolicie należy poczynić zmiany, które uzależniają wszystkie czynności należące do zarządzania kontem użytkownika generycznego przetransferować by podlegały walidacji oraz wykonaniu przez mikroserwis. Na schemacie pojawia się zmiana nazewnictwa z “Gniazdo PHP” na “Monolit”, co od teraz symbolizować będzie całość logiki, ponieważ warstwa infrastruktury już nie jest istotna w tym momencie. Dodatkowo w oprogramowaniu pośredniczącym w zapytaniu między użytkownikiem a systemem - nginx należy poczynić zmianę konfiguracji, tak by ta przekazywała zapytania o użytkownika bezpośrednio do odpowiedniego kontenera z mikroserwisem w wirtualnej sieci Docker. W systemie do zarządzania relacyjną bazą danych należy utworzyć bazę danych odpowiadającą za tylko i wyłącznie dane wymagane do utrzymania w systemie przez mikroserwis użytkownika. Wyszczególniona powyżej zmiana została odzwierciedlona na rysunku~\ref{fig:migration-phase-3}.

    \begin{figure}[ht]
        \centering
        \includegraphics[width=\textwidth]{includes/images/migration-phase-3.png}
        \captionvspace
        \caption{Schemat architektury systemu do internetowego wspomagania pacjenta i lekarza w trzeciej fazie}
        \label{fig:migration-phase-3}
    \end{figure}


    \section{Faza czwarta}
    Faza czwarta migracji aplikacji zawiera w sobie dalszą modyfikację systemu. Początek zmian należy zawrzeć w konfiguracji oprogramowania nginx, gdzie należy przekierować cały ruch odpowiadający za obsługę użytkownika do mikroserwisu pacjenta. Dodatkowo utworzenie nowej bazy danych w systemie do zarządzania bazą danych MySQL. Niniejsza zmiana zwizualizowna została na rysunku~\ref{fig:migration-phase-4}.

    \begin{figure}[ht]
        \centering
        \includegraphics[width=\textwidth]{includes/images/migration-phase-4.png}
        \captionvspace
        \caption{Schemat architektury systemu do internetowego wspomagania pacjenta i lekarza w czwartej fazie}
        \label{fig:migration-phase-4}
    \end{figure}


    \section{Faza piąta}
    Kolejną fazą migracji jest część piąta, a ta będzie analogiczna do fazy czwartej. W jej skład będzie wchodzić utworzenie mikroserwisu dla pracownika recepcji oraz przekierowanie całego ruchu odpowiadającego za obsługę tej domeny do nowego mikroserwisu. Faza zostanie zakończona utworzeniem nowej bazy danych. Ta zmiana została przedstawiona na rysunku~\ref{fig:migration-phase-5}.

    \begin{figure}[ht]
        \centering
        \includegraphics[width=\textwidth]{includes/images/migration-phase-5.png}
        \captionvspace
        \caption{Schemat architektury systemu do internetowego wspomagania pacjenta i lekarza w piątej fazie}
        \label{fig:migration-phase-5}
    \end{figure}


    \section{Faza szósta}
    W fazie szóstej dodano już ostatni mikroserwis odpowiadający za domeny związane z pracownikami lub użytkownikami. Utworzono zmiany w konfiguracji oprogramowania nginx oraz dodano bazę danych dla pracownika administracji. Zmiany zostały przedstawione na rysunku~\ref{fig:migration-phase-6}.

    \begin{figure}[ht]
        \centering
        \includegraphics[width=\textwidth]{includes/images/migration-phase-6.png}
        \captionvspace
        \caption{Schemat architektury systemu do internetowego wspomagania pacjenta i lekarza w szóstej fazie}
        \label{fig:migration-phase-6}
    \end{figure}


    \section{Faza siódma}
    W fazie siódmej migracji dodano mikroserwis odpowiadający za obsługę wizyt. Proces technologiczny wygląda analogicznie do wcześniejszych faz. Wykonano zmianę konfiguracji oprogramowania nginx oraz utworzono nową bazę danych. Zmiany widoczne na rysunku~\ref{fig:migration-phase-7}.

    \begin{figure}[ht]
        \centering
        \includegraphics[width=\textwidth]{includes/images/migration-phase-7.png}
        \captionvspace
        \caption{Schemat architektury systemu do internetowego wspomagania pacjenta i lekarza w siódmej fazie}
        \label{fig:migration-phase-7}
    \end{figure}


    \section{Faza ósma}
    Ostatnia faza migracji to uruchomienie oprogramowania RabbitMQ, które będzie brokerem komunikatów portu AMQP. Dodatkowo uruchomiono serwis wysyłki maili. Na schemacie nastąpiło usunięcie części systemu określonej jako monolit, a to dlatego, że cały system w obecnym stanie jest już niezależny od jednej całości. Całość przedstawiona na rysunku~\ref{fig:migration-phase-8}.

    \begin{figure}[ht]
        \centering
        \includegraphics[width=\textwidth]{includes/images/migration-phase-8.png}
        \captionvspace
        \caption{Schemat architektury systemu do internetowego wspomagania pacjenta i lekarza w ósmej fazie}
        \label{fig:migration-phase-8}
    \end{figure}


    \section{Podsumowanie przebiegu migracji}
    Proces stopniowego ewoluowania projektu monolitycznego w projekt mikroserwisowowy dla systemu do internetowego wspomagania pacjenta i lekarza, jest procesem skomplikowanym. Czas poświęcony na dokonanie takiej migracji to w przybliżeniu 112h. Proces należał jednak do łatwiejszych przez wzgląd na niekomercyjne wykorzystanie systemu, a co za sobą wprowadza mniejsze ryzyko utraty danych lub poważnej awarii całej aplikacji.


    \chapter{Bezpieczeństwo}


    \section{Poziom zabezpieczeń przed migracją}
    Niniejszy rozdział przedstawia stan poziomu zabezpieczeń przed migracją autora na system mikroserwisowy. Aspektem cechującym wszystkie systemy oparte o architekturę monolityczną jest fakt, że w przypadku awarii jednej usługi wewnątrz monolitu, niesie to ryzyko rozpropagowania awarii na cały system i zaburzenie jego działania. Ocena poziomu zabezpieczeń będzie krytyczna, ale złagodzona przez fakt, iż system nie działa w zakresie komercyjnym.

    System do zarządzania bazą danych MySQL posiadał wyeksponowany na cały internet port, po którym potencjalny atakujący mógł przeprowadzić atak typu Brute Force w celu uzyskania dostępu do funkcji administracyjnych i potencjalnie pozyskanie danych przetrzymywanych w bazie danych. Atak typu Brute Force polega na próbie odgadnięcia hasła przez generowanie wszystkich możliwych kombinacji ciągu znaków. W teorii można złamać w ten sposób każde hasło \cite{brute.forcre}. Problem ten można rozwiązać na wiele sposobów, jednak autor tej pracy dyplomowej zastosuje ukrycie bazy danych w prywatnej sieci wewnętrznej, a dostęp do narzędzi administratorskich będzie się odbywać poprzez klucz SSH generowane przy pomocy algorytmów kryptograficznych takich jak RSA lub ECDSA.

    Modernizowana aplikacja omawiana w tej pracy dyplomowej pierwotnie uruchomiona była na porcie HTTP 80, bez dodatkowych zabezpieczeń w postaci ważnego certyfikatu klucza publicznego. Certyfikat ten to informacja o kluczu publicznym podmiotu, która podpisana jest przez zaufaną stronę trzecią i jest niemożliwa do podrobienia \cite{pkn.ssl}. Wynikało to z faktu, iż dla celów akademickich nie istotnym było utworzenie owego certyfikatu i zadbanie o komunikację użytkownika poprzez szyfrowanych kanał na porcie HTTP 443. Kolejnym krokiem do przeprowadzenia odpowiedniej migracji mającej na celu uwzględnienie cyberbezpieczeństwo systemu jest dokonanie zmiany na komunikację szyfrowaną.

    Ostateczne problemy wynikające z pomniejszego zaniedbania autora to fakt, iż system nie otrzymywał regularnych aktualizacji zabezpieczeń. W tym celu by zwiększyć poziom bezpieczeństwa należy uruchomić procedurę aktualizacji bibliotek zewnętrznych wykorzystywanych przez framework (ang. zestaw narzędzi) Symfony oraz także przeprowadzić aktualizację wersji interpretera języka programowania PHP. Ten sam fakt dotyczy również bibliotek, z których korzystała warstwa widoku pod postacią zestawu narzędzi Vue.js.


    \section{Bezpieczeństwo architektury chmurowej}
    W dokumentacji narzędzia Amazon VPC podkreśla się, że dostarcza ono kluczowe rozwiązanie z punktu widzenia bezpieczeństwa każdego systemu opartego na obliczeniach chmurowych — izolację zasobów. Jest to możliwe dzięki logicznej izolacji wirtualnej sieci, którą definiuje użytkownik AWS. Narzędzie to oferuje również funkcje takie jak kreacja własnego zakresu adresów IP, tworzenie podsieci, konfiguracja tabel routingu oraz bramek internetowych, co pozwala na maksymalizację poziomu bezpieczeństwa i dostosowanie infrastruktury sieciowej do specyficznych potrzeb systemu.
    Wyżej wymienione funkcje udostępnione przez Amazon w narzędziu VPC zostały częściowo wykorzystane przez autora niniejszej pracy dyplomowej. Obecne rozwiązania sieciowe są znacznie bardziej dostosowane do wymagań systemu do internetowego wspomagania pacjenta i lekarza w porównaniu z pierwotną konfiguracją.

    Aspekt kontroli dostępu do zasobów w chmurze jest zarządzany za pomocą narzędzia Amazon IAM. Jest to podstawowy wybór w zakresie konfiguracji kontroli dostępu, od którego zależy określanie uprawnień. Na pierwszy rzut oka IAM może wydawać się skomplikowany, jednak po zapoznaniu się z jego dokumentacją i poradnikami, ukazuje swoją prostotę oraz przewagę, jaką zyskuje użytkownik konfigurujący zasoby swojego projektu z uwzględnieniem bezpieczeństwa. IAM oferuje bardzo szerokie spektrum konfiguracji — od ogólnych ustawień po najbardziej szczegółowe, co pozwala precyzyjnie określić uprawnienia osób, które potrzebują dokonać zmian.

    Jedną z zalet tego rozwiązania jest możliwość tworzenia tymczasowych haseł dostępu, co umożliwia dynamiczne uruchamianie poszczególnych części infrastruktury lub automatyzowanie procesu wdrożenia, jeśli użytkownik konfigurujący dąży do utworzenia oprogramowania w podejściu IaaS (Infrastructure as a Service).

    Dodatkowo narzędzie to posiada wbudowany program do analizy dostępu oraz walidacji skonfigurowanej polityki uprawnień. Program ten zamieszcza porady, jak efektywnie zmniejszyć uprawnienia, aby nie były one nadmierne, a jedynie pozwalały na wykonanie niezbędnych zadań \cite{aws.iam}.


    \section{Bezpieczeństwo komunikacji}
    Naturalnym aspektem wymiany danych pomiędzy poszczególnymi komponentami systemu modernizowanego przez autora pracy jest uwzględnienie szyfrowanej komunikacji. Efekt ten uzyskuje się poprzez wykorzystanie protokołów TLS/SSL, które zapewniają poufność i integralność przesyłanych danych. TLS wykorzystuje szyfrowanie asymetryczne.

    Szyfrowanie asymetryczne to rodzaj kryptografii, w którym jeden z kluczy jest kluczem publicznym \cite{nature.tls.ssl}. Dowolny użytkownik może użyć tego klucza do zaszyfrowania wiadomości, jednak tylko posiadacz drugiego, prywatnego klucza może ją odszyfrować. W uproszczeniu, tak przebiega komunikacja z wykorzystaniem protokołu TLS/SSL.

    Z perspektywy modelu OSI, będącego standardem komunikacji komputerowej zaproponowanym przez ISO, TLS odgrywa kluczową rolę w warstwie prezentacji, co pozytywnie wpływa na zabezpieczenie warstwy najwyższej — warstwy aplikacji. W przypadku systemu modernizowanego przez autora, jest to warstwa związana z protokołem HTTP.

    W gamie usług AWS, narzędziem zarządzającym i dostarczającym darmowe certyfikaty do szyfrowanej komunikacji jest AWS Certificate Manager (menedżer certyfikatów AWS). Usługa ta ułatwia centralne zarządzanie certyfikatami, zarówno tymi wygenerowanymi przez Amazon, jak i dostarczonymi z zewnątrz. Dodatkowo AWS Certificate Manager bezpiecznie przechowuje certyfikaty, co zwalnia użytkownika korzystającego z usług AWS z odpowiedzialności za weryfikację potencjalnych luk w zabezpieczeniach \cite{aws.acm}.


    \section{Zarządzanie danymi}
    Przechowywanie danych w zmodernizowanym systemie do wspomagania pacjenta i lekarza będzie zarządzane przez usługę Amazon RDS. Usługa ta oferuje szyfrowanie danych zarówno w spoczynku, jak i podczas wymiany, co pozytywnie wpływa na poziom zabezpieczeń systemu. Szyfrowanie danych w spoczynku realizowane jest przy użyciu kluczy dostarczonych przez użytkownika. Dane są szyfrowane niezależnie od tego, czy są aktywnie używane, czy przechowywane w kopiach zapasowych bazy danych \cite{aws.rds}.

    Mając na uwadze dobro użytkowników systemu, a także regulacje Unii Europejskiej, szczególną uwagę należy zwrócić na zabezpieczenie danych zgodnie z rozporządzeniem o ochronie danych osobowych (RODO). Celem tego rozporządzenia jest harmonizacja prawa w ramach państw członkowskich Unii Europejskiej oraz umożliwienie swobodnego przepływu danych osobowych \cite{giodo}. Amazon potwierdza, że usługi tej firmy są zgodne z RODO, zatem po odpowiednim skonfigurowaniu systemu aplikacja uruchomiona w chmurze obliczeniowej AWS będzie również zgodna z tym rozporządzeniem \cite{aws.gdpr}.


    \section{Bezpieczeństwo aplikacji}
    W niniejszym rozdziale omówione zostaną aspekty wewnętrzne bezpieczeństwa aplikacji, między innymi wykorzystanie narzędzia Dependabot, dostarczanego przez firmę GitHub. Narzędzie to oferuje trzy główne funkcjonalności:

    \begin{itemize}
        \item alerty informujące o podatnościach występujących w aplikacji,
        \item automatyczne tworzenie w repozytorium kodu Pull Request (ang. prośba o dodanie zmian) z usprawnieniami mającymi na celu likwidację podatności, jeśli jest to możliwe,
        \item automatyczne tworzenie w repozytorium kodu próśb o dodanie zmian z aktualizacją wersji, nawet jeśli nie wykryto podatności.
    \end{itemize}

    Aby skorzystać z tego narzędzia, konieczne jest posiadanie repozytorium na platformie GitHub. Autor tej pracy od początku tworzenia projektu korzysta z tej platformy do przechowywania kodu. Po utworzeniu repozytorium należy przejść do jego ustawień, a następnie w zakładce „Security” otworzyć sekcję „Code security and analysis”, gdzie znajduje się przełącznik umożliwiający uruchomienie Dependabota. Po włączeniu narzędzia, Dependabot automatycznie analizuje zależności i strukturę kodu, informując o potencjalnych problemach oraz sugerując ich rozwiązania \cite{github.dependabot}. Na rysunku~\ref{fig:github-dependabot} przedstawiony przykład komunikatu, który dostarcza GitHub Dependabot.

    \begin{figure}[ht]
        \centering
        \includegraphics[width=\textwidth]{includes/images/example-dependabot-notification.png}
        \captionvspace
        \caption{Zrzut ekranu przedstawiający przykładowy komunikat dotyczący podatności dla wersji systemu do wspomagania pacjenta i lekarza przed migracją}
        \label{fig:github-dependabot}
    \end{figure}

    Autor, prowadząc rozpoznanie w bazie kodu aplikacji, przeprowadził zarówno analizę pasywną, jak i aktywną. Pasywny rekonesans polegał na zgłębieniu wymagań funkcjonalnych systemu, aby odświeżyć pamięć o czynnościach, które system powinien zapewniać. Dodatkowo, w ramach pasywnej analizy, autor sprawdził domenę sdiwpil.com w rejestrze DNS i zweryfikował informacje oraz jakość zabezpieczeń.

    Jednym z mankamentów, który został zidentyfikowany po wprowadzonych przez autora modyfikacjach, jest fakt, że system posiada jednego dostawcę DNS, którym jest Amazon. W celu uniezależnienia systemu od ewentualnej awarii AWS, autor rozważa możliwość rozlokowania wpisów DNS u różnych dostawców. Mimo to, autor wierzy, że Amazon dysponuje szeregiem zabezpieczeń mających na celu zapewnienie redundancji swoich zasobów, co daje pewność, że klienci mogą czuć się bezpiecznie, powierzając każdy detal swojego systemu w rękach specjalistów AWS.

    Dalsza część analizy obejmowała rekonesans aktywny, który polegał na zidentyfikowaniu punktów końcowych API, wersji oprogramowania serwera, potencjalnych informacji o modułach PHP, komponentów JavaScript oraz innych szczegółów związanych z architekturą aplikacji.

    Aplikacja budowana i modernizowana przez autora nie posiada dokumentacji API, ponieważ jej API jest skierowane na współpracę z warstwą widoku, a nie na udostępnianie usług zewnętrznym klientom. Weryfikacja działania API może być zatem przeprowadzona poprzez symulację komunikacji między warstwą aplikacji a serwerem. Informacje, które można uzyskać bezpośrednio po uruchomieniu strony, obejmują punkty końcowe API związane z rejestracją i logowaniem użytkowników, zasoby statyczne, takie jak zdjęcia, pliki widoku, pliki HTML, CSS i JavaScript. Wymienione punkty końcowe są otwartymi i dostępnymi ścieżkami, które nie wymagają uwierzytelnienia użytkownika. Po zalogowaniu się na konta testowe dla poszczególnych aktorów systemu, ukazuje się pełne spektrum dostępnych ścieżek w systemie, które zostały zamieszczone w tabeli~\ref{tab:internal_api}.

    \begin{longtable}{|>{\raggedright\arraybackslash}p{3cm}|>{\raggedright\arraybackslash}p{2cm}|>{\raggedright\arraybackslash}p{5cm}|>{\raggedright\arraybackslash}p{3.5cm}|}
        \caption{Opis punktów końcowych systemu do internetowego wspomagania pacjenta i lekarza} % Podpis tabeli
        \label{tab:internal_api} \\
        \hline
        \textbf{Punkt końcowy API}      & \textbf{Dozwolona metoda HTTP} & \textbf{Krótki opis punktu} & \textbf{Aktor posiadający dostęp} \\ \hline
        \endfirsthead

        \hline
        \textbf{Punkt końcowy API}      & \textbf{Dozwolona metoda HTTP} & \textbf{Krótki opis punktu} & \textbf{Aktor posiadający dostęp} \\ \hline
        \endhead

        \hline
        \endfoot

        \hline
        \endlastfoot

        /login/                         & POST                           & Przesłanie danych logowania i pobieranie tokena uwierzytelniającego. & użytkownik niezalogowany \\ \hline
        /register/                      & POST                           & Przesłanie danych do utworzenia konta i pobieranie tokena uwierzytelniającego. & użytkownik niezalogowany \\ \hline
        /patient/ settings/get/         & GET                            & Pobranie ustawień konta pacjenta.                                              & pacjent                                                      \\ \hline
        /patient/ settings/update/      & PUT i PATCH                    & Zapisanie ustawień konta pacjenta.                                             & pacjent                                                      \\ \hline
        /appointments/                  & GET                            & Pobranie listy umówionych wizyt.                                               & pacjent, lekarz, pracownik recepcji, pracownik administracji \\ \hline
        /appointments/ create/          & POST                           & Utworzenie nowej wizyty.                                                       & pacjent, lekarz, pracownik recepcji, pracownik administracji \\ \hline
        /appointments/ get/{id}         & GET                            & Pobranie szczegółów jednej wizyty.                                             & pacjent, lekarz, pracownik recepcji, pracownik administracji \\ \hline
        /appointments/ update/{id}      & PUT i PATCH                    & Aktualizacja danych wizyty.                                                    & pacjent, lekarz, pracownik recepcji, pracownik administracji \\ \hline
        /appointments/ delete/{id}      & DELETE                         & Usunięcie danych wizyty.                                                       & pacjent, lekarz, pracownik recepcji, pracownik administracji \\ \hline
        /doctor/ settings/get/          & GET                            & Pobranie ustawień konta lekarza.                                               & lekarz                                                       \\ \hline
        /doctor/ settings/update/       & PUT i PATCH                    & Zapisanie ustawień konta lekarza.                                              & lekarz                                                       \\ \hline
        /receptionist/ settings/get/    & GET                            & Pobranie ustawień konta pracownika recepcji.                                   & pracownik recepcji                                           \\ \hline
        /receptionist/ settings/update/ & PUT i PATCH                    & Zapisanie ustawień konta pracownika recepcji. & pracownik recepcji \\ \hline
        /admin/ settings/get/           & GET                            & Pobranie ustawień konta pracownika administracji.                              & pracownik administracji                                      \\ \hline
        /admin/ settings/update/        & PUT i PATCH                    & Zapisanie ustawień konta pracownika administracji. & pracownik administracji \\ \hline

    \end{longtable}

    Autor, kontynuując aktywny rekonesans systemu do internetowego wspomagania pacjenta i lekarza, uzyskał informacje dotyczące serwera pośredniczącego w zapytaniach oraz wersji protokołu HTTP używanej w komunikacji. W celu zdobycia tych danych, autor skorzystał z wbudowanego narzędzia w programie PHPStorm, które umożliwia wykonywanie zapytań HTTP za pośrednictwem swojego interfejsu. Odpowiedź na pytanie, jaki serwer jest pośrednikiem, została przedstawiona na rysunku~\ref{fig:sdiwpil-response}, a jest to nginx.

    \begin{figure}[ht]
        \centering
        \includegraphics[width=\textwidth]{includes/images/example-response-from-sdiwpil.png}
        \captionvspace
        \caption{Zrzut ekranu przedstawiający przykładową odpowiedź od jednego z mikroserwisów systemu do internetowego wspomagania pacjenta i lekarza}
        \label{fig:sdiwpil-response}
    \end{figure}

    Podążając za dobrymi praktykami, autor ukrył w konfiguracji nginx informacje takie jak wersja oprogramowania, aby potencjalny atakujący nie miał ułatwionego zadania poprzez wyszukiwanie luk zabezpieczeń specyficznych dla danej wersji. Dodatkowo został ukryty nagłówek X-Powered-By informujący o wersji PHP 8.3, aby uniemożliwić atakującemu celowanie w konkretne podatności związane z interpreterem PHP lub samym językiem programowania.

    Kolejnym krokiem aktywnego rekonesansu będzie analiza plików źródłowych aplikacji. Pierwszym analizowanym plikiem będzie plik konfiguracyjny nginx.

    Plik ten składa się z kilku kluczowych bloków. Jednak przed analizą samych bloków warto zwrócić uwagę, że konfiguracja zaczyna się od ustawienia \texttt{worker\_processes 4;}, co oznacza, że nginx uruchamia cztery procesy wykonawcze. Cały plik konfiguracyjny został przedstawiony na rysunku~\ref{fig:nginx-config}.

    \begin{figure}[ht]
        \centering
        \includegraphics[width=0.48\textwidth]{includes/images/nginx-configuration.png}
        \captionvspace
        \caption{Zrzut ekranu przedstawiający konfigurację narzędzia nginx}
        \label{fig:nginx-config}
    \end{figure}

    Pierwszym blokiem w pliku konfiguracyjnym nginx jest blok events (ang. zdarzenia), który zawiera ustawienie \texttt{worker\_connections 1024;}. Ustawienie to określa, ile maksymalnie połączeń może obsłużyć pojedynczy proces \cite{nginx.worker.processes}.

    Sekcja http, z punktu widzenia cyberbezpieczeństwa systemu, zawiera ustawienia takie jak \texttt{limit\_req\_zone \$binary\_remote\_addr zone=one:10m rate=1r/s;} \cite{nginx.limit.req.zone}, które definiują strefę ograniczenia liczby żądań. To ustawienie wykorzystuje adres IP klienta jako klucz (\texttt{\$binary\_remote\_addr}) i tworzy strefę o nazwie „one” z rozmiarem pamięci ustawionym na 10 MB, co pozwala na 1 żądanie na sekundę z jednego adresu IP. Jest to szczególnie istotne z punktu widzenia ochrony przed potencjalnym atakiem DDoS (Distributed Denial of Service, rozproszona odmowa usługi), który ma na celu zablokowanie maszyny utrzymującej system w sieci \cite{cisa.ddos}. Konfiguracja ta odrzuca nadmierną ilość zapytań z tego samego adresu IP, jeśli przekraczają one ustalone limity, co pomaga chronić system przed przeciążeniem i utratą dostępności.

    Dalsze sekcje pliku konfiguracyjnego nginx nie zawierają specjalnych ustawień, które znacząco wpłynęłyby na bezpieczeństwo zewnętrzne. Niemniej jednak warto wspomnieć o potencjalnych usprawnieniach, które mogą zostać wprowadzone, takich jak:

    \begin{itemize}
        \item dodanie większej liczby serwerów obsługujących dany upstream,
        \item rozszerzona konfiguracja logów, która może być rozbudowana o specyficzne ustawienia dla bloków location (lokalizacja),
        \item wprowadzenie wewnętrznej komunikacji za pośrednictwem portu HTTPS.
    \end{itemize}

    Kolejnym obszarem analizy są zależności używane w kodzie źródłowym systemu. Dla celów analitycznych tej pracy zostanie przeanalizowany jeden z plików composer.json, który określa zależności w poszczególnych mikroserwisach. Taka analiza jest uzasadniona, ponieważ system opiera się na tym samym zestawie narzędzi w każdym mikroserwisie. Mikroserwisem, którego plik composer.json zostanie poddany audytowi, jest mikroserwis autoryzacji.

    Sekcją szczególnie wrażliwą z punktu widzenia bezpieczeństwa aplikacji jest sekcja require, która zawiera obiekt składający się z par klucz-wartość, gdzie klucz to nazwa zewnętrznej biblioteki, a wartość to wersja wykorzystywanej biblioteki. Cała sekcja jest widoczna na rysunku~\ref{fig:composer-require}. W dniu 23 sierpnia 2024 roku, po uruchomieniu narzędzia audit, wbudowanego w program Composer, można sprawdzić, czy są dostępne sugestie dotyczące zabezpieczeń. W przypadku zmodernizowanego przez autora projektu nie występują żadne zalecenia, co wskazuje na wysoki poziom bezpieczeństwa projektu w tym zakresiea, a rezultat przedstawiony jest na rysunku~\ref{fig:composer-audit}.

    \begin{figure}[ht]
        \centering
        \includegraphics[width=\textwidth]{includes/images/composer-require-section.png}
        \captionvspace
        \caption{Zrzut ekranu przedstawiający sekcję require z pliku composer.json mikroserwisu autoryzacji}
        \label{fig:composer-require}
    \end{figure}

    \begin{figure}[ht]
        \centering
        \includegraphics[width=\textwidth]{includes/images/composer-audit.png}
        \captionvspace
        \caption{Zrzut ekranu przedstawiający wynik audytu zależności w pliku composer.json mikroserwisu autoryzacji}
        \label{fig:composer-audit}
    \end{figure}

    Aplikacja do internetowego wspomagania pacjenta i lekarza wykorzystuje Doctrine jako pośredniczący mechanizm do połączenia z bazą danych. Doctrine zapewnia liczne ułatwienia, takie jak mapowanie obiektów na encje, mechanizmy usprawniające efektywność zapytań wykonywanych do bazy danych oraz wiele innych funkcji.

    Jednym z kluczowych aspektów bezpieczeństwa, które Doctrine zapewnia, jest ochrona przed atakami typu SQL injection. Dzięki mechanizmom wiązania dynamicznych parametrów zapytań, Doctrine znacząco minimalizuje ryzyko tego rodzaju ataków \cite{doctrine.orm}. Tylko świadome zrezygnowanie z korzystania z jego mechanizmów mogłoby otworzyć drogę do przeprowadzenia ataku, jednak jest to proces trudny do wykonania, gdyż wymagałoby to celowego wyłączenia wszystkich zabezpieczeń oferowanych przez Doctrine. Dlatego autor pracy ocenia poziom odporności na ataki SQL injection jako wysoki.

    Sama dokumentacja Doctrine przedstawia przykłady, w jaki sposób unikać niebezpiecznych praktyk, które mogłyby prowadzić do tego rodzaju ataków \cite{doctrine.security}. Autor podkreśla, że w zmodernizowanym systemie do wspomagania pacjenta i lekarza, jak również w jego pierwotnej wersji, nie stosowano podejścia, które mogłoby narażać system na atak typu SQL injection.

    \section{Podsumowanie}
    Bezpieczeństwo każdego systemu działającego w internecie w dużej mierze zależy od uwagi poświęconej podczas jego budowy oraz świadomości inżyniera implementującego konkretną architekturę. Niemniej jednak, obecnie dostępne na rynku rozwiązania i frameworki skutecznie chronią mniej doświadczonych użytkowników przed podstawowymi formami podatności. Tylko świadome odrzucenie tych mechanizmów może narazić system na ryzyko.

    Oczywiście, powyższe stwierdzenie nie wyklucza możliwości odkrycia luk w innych obszarach, niż te omówione w rozdziale. Autor pracy wyraża chęć dalszego pogłębiania wiedzy zdobytej podczas analizy przeprowadzonej na potrzeby niniejszej pracy dyplomowej, mając świadomość, że tylko ciągły rozwój w tym zakresie oraz skuteczne wykorzystanie dostępnych narzędzi są kluczem do skutecznej ochrony systemu.


    \bibliographystyle{plain}
    \bibliography{bibliography.bib}
    \listoffigures
    \listoftables
\end{document}
