\documentclass[12pt,twoside]{book}
\usepackage[a4paper,inner=3.5cm,outer=2.5cm,top=2.5cm,bottom=2.5cm]{geometry}
\usepackage{fontspec}
\setmainfont{Times New Roman}
\usepackage{setspace}
\setstretch{1.5}
\usepackage{fancyhdr}
\pagestyle{fancy}
\fancyhf{}
\fancyfoot[CE,CO]{\thepage}
\renewcommand{\headrulewidth}{0pt}
\fancypagestyle{plain}{%
  \fancyhf{}%
  \renewcommand{\headrulewidth}{0pt}%
  \fancyfoot[CE,CO]{\thepage}%
}
\usepackage{titlesec}
\titleformat{\chapter}[display]
  {\normalfont\huge\bfseries}
  {\chaptertitlename\ \thechapter}{20pt}{\Huge}
\titlespacing*{\chapter}{0pt}{12pt}{6pt}
\usepackage{lipsum}
\usepackage{pdfpages}
\usepackage{tocloft}
\renewcommand{\contentsname}{Spis treści}
\renewcommand{\cftchapleader}{\cftdotfill{\cftdotsep}}

\setlength{\parindent}{1.25cm}

\begin{document}

\frontmatter
\thispagestyle{empty}
\begin{titlepage}
\includepdf{includes/pdf/strona_tytułowa.pdf}
\end{titlepage}

\cleardoublepage
% Tutaj umieść swoją stronę ze spisem treści
\tableofcontents

\mainmatter
% Tutaj zaczyna się główna treść pracy
\chapter{Wprowadzenie}
\lipsum[1-4] % Generuje przykładowy tekst

\end{document}
